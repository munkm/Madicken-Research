\documentclass{report}
\usepackage[hmargin=3cm,vmargin=3.5cm]{geometry}
\usepackage{multirow}
\usepackage{outlines}
\begin{document}
\title{Automated Weight Windows for Global MC \\
         Transport Calculations}
\author{Cooper & Larsen}
\date{\today}
% \maketitle
\setcounter{secnumdepth}{0}
\section{Automated Weight Windows for Global MC Transport Calculations}
\subsection{Abstract}
\begin{outline}
 \1 Method provides flux solution over entire system (not just at detectors)
 \1 Uses weight window
    \2 Distributes MC particles uniformly throughout system
      \3 All subregions are adequately sampled
      \3 Particle weights are controlled, even far from the source. 
    \2 Constructed from forward transport solution
      \3 More appropriate for global problems. 
      \3 Does not use adjoint. 
    \2 Can be used with Eddington factors from deterministic solutions to update weight windows for MC. 
\end{outline}
\subsection{Introduction}
\begin{outline}
  \1 Previous methodology for VR was applied to local flux solutions with detectors. 
  \1 Problems with global flux solution
    \2 Only have survival biasing as a form of VR
      \3 Large statistical errors occur away from source regions. 
      \3 long run times
  \1 New method
    \2 Is general, can be applied to neutrons, photons, charged-particles
    \2 Especially adept for neutrons with deep penetration (where flux varies by OOM)
      \3 Classically done with survival biasing and weights, but ends up with large variances. 
    \2 Utilizes forward transport problem generated weight window
      \3 Not adjoint. 
      \3 Distributes the MC particles uniformly throughout the system. 
      \3 Improves the point wise FOM. 
      \3 Generated with diffusion solution
        \4 Diffusion is faster than other deterministic methods
        \4 MC is slow, and errors in MC can bias weights away from interesting areas. 
        \4 Not Sn or PN, as accuracy is not required to generate WWs, they are expensive, and Sn are susceptible to ray effects. 
    \2 Quasi-diffusion method
      \3 Used to improve accuracy of initial diffusion WW. 
        \4 MC process can be used to obtain estimates of Eddington factors—> can be used in quasi-diffusion method to improve flux estimates—> improves WW. 
      \3 Better accuracy than traditional diffusion, but faster than Sn or Pn. 
    \2 Two types of Weight Windows 
      \3 Isotropic
        \4 Computed only from scalar fluxes
        \4 Russian Roulette and splitting performed independently of particle direction and flight
      \3 Angular
        \4 Employs scatter fluxes and currents
        \4 Russian Roulette and splitting performed based of particle direction and flight
        \4 Implemented in MCNP using AVATAR method. 
\end{outline}
\end{document}
