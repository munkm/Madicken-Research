\documentclass{report}
\usepackage[hmargin=3cm,vmargin=3.5cm]{geometry}
\usepackage{multirow}
\usepackage{outlines}
\begin{document}
\title{AVATAR- Automatic Variance and Time \\
         of Analysis Reduction}
\author{asdf}
\date{\today}
% \maketitle
\setcounter{secnumdepth}{0}
\section{AVATAR - Automatic Variance and Time of Analysis Reduction}
\subsection{Abstract}
\begin{outline}
  \1 Avatar —> accessed through Justine—> invokes THREEDANT
  \1 Performs 3-dimensional deterministic adjoint calculation 
  \1 Mesh independent of MC geometry
  \1 calculates weight windows
  \1 Can be done without user manipulation of data
\end{outline}
\subsection{Introduction}
\begin{outline}
  \1 For problems with complex geometries and with a large problem area, Monte Carlo can be an issue. The desired quantity is OOM smaller than the source. Tally of particles contributing to return signal is small, and unreliable. 
  \1 Variance Reduction Techniques
    \2 Weight windows
      \3 depend on importance functions. 
      \3 bias the particles towards locations that contribute to the tally.
      \3 Requires time, experience, and intuition. 
    \2 Solution of Adjoint
      \3 is recognized as an importance function for the regular forward calculation of NTE. 
        \4 Is computationally expensive to calculate exactly. 
      \3 Less  precise calculation provides importance function that can be used to accelerate MC calculation. 
        \4 do this with few energy groups and low angular resolutionn
  \1 AVATAR
    \2 automatically accelerates MC calculations based on space, energy and angle. 
    \2  User sets up problem using Justine (the GUI)
    \2 Justine runs Avatar, 
      \3 superimposes a weight window mesh 
      \3 runs THREEDANT in adjoint mode
      \3 constructs weight windows from adjoint solution
      \3 runs MCNP
    \2 User has option to intervene
\end{outline}
\subsection{The Adjoint Flux}
\begin{outline}
  \1 In adjoint calculation, particles emanate from forward detectors—and transport backwards. 
  \1 Can be calculated two ways
    \2 Stochastically with MC
      \3 provides too much detail and costs just as much as a forward calculation. 
    \2 Determinstically with discrete ordinates transport or diffusion
      \3 with low resolution gives importances sufficient to speed up calculation.
\end{outline}
\subsection{AVATAR}
\begin{outline}
  \1 Mesh Description
    \2 Can be in rectangular or cylindrical coordinate system
    \2 Coarse mesh with fine mesh intervals. 
      \3 Material of each mesh element is material at center of mesh element. 
  \1 Automatic Mesh Generation
    \2 Bounding box in each direction divided into coarse elements. 
      \3 Uniform mesh = dividing boundary box uniformly in each direction
      \3 “Smart” mesh = subdividing boundary box into smaller boundary boxes that bound each MCNP cell
        \4 Justine defaults to “smart” mesh. 
    \2 Coarse mesh may be subdivided uniformly or logarithmically into fine meshes
      \3 defuaults to uniform, which is also slightly faster. 
    \2 User can intervene and add or delete mesh lines. 
  \1 Determinstic Adjoint Calculation 
    \2 Executed with THREEDANT—performs deterministic discrete-ordinates transport calculation 
      \3 AVATAR executes a lower resolution calculation with threedant. 
        \4 quadrature of S4
        \4 P2 legendre scattering moment expansion
        \4 three energy groups
    \2 Produces a scalar adjoint flux and average current for 
      \3 each mesh element
      \3 each energy group. 
  \1 Weight Windows
    \2 In each mesh element and each group, the scalar adjoint fluxes are inverted to get lower WW boundary. 
      \3 WW is normalized
      \3 Regions with high flux = high importance = low weight. 
  \1 Angle-Dependent Weight Windows
  \1 Mean Free Path Sampling
    \2 Particles may arrive in a location where their weights are significantly outside of the local weight window, which increases the variance in particle weights, which increases the tally variance. 
    \2 Avatar protects against this by applying weight windows at least every mean free path that the particles travel. 
\end{outline}
\end{document}
