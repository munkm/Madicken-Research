\documentclass{report}
\usepackage[hmargin=3cm,vmargin=3.5cm]{geometry}
\usepackage{multirow}
\usepackage{outlines}
\begin{document}
\title{The Quasi-DIffusion method for solving transport\\
         problems in planar and spherical geometries}
\author{Miften and Larsen}
\date{\today}
% \maketitle
\setcounter{secnumdepth}{0}
\section{The Quasi-Diffusion method for solving transport problems in planar and spherical geometries}
\subsection{Abstract}
\begin{outline}
 \1 Quasidiffusion is used to solve linear transport problems, and converges rapidly with high accuracy
   \2 using a transport sweep
   \2 and a diffusion calculation 
 \1 more complex than Source Iteration (SI) method
 \1 pros:
   \2 good for optically thick problems 
   \2 good for problems with a scattering ratio close to unity
 \1 cons: 
   \2 every scalar flux must be positive at each point in the system
   \2 formulation of diffusion BCs for optimization not obvious
\end{outline}
\subsection{Introduction}
\begin{outline}
 \1 Quasidiffusion Advantages:
   \2 allows flexibility in the choices of differencing schemes for transport and diffusion
   \2 may be effective for multidimensional problems with non rectangular meshes
 \1 Disadvantages:
   \2 angular flux iterates must be positive at each point in the system
     \3 otherwise eddington factors can become negative or infinite
       \4 results in unstable diffusion
   \2 optimizing diffusion BCs to optimize accuracy and speed of QD not obvious
   \2 produces two different solutions separated by truncation error
     \3 (1) scalar flux from the transport calculation
     \3 (2) scalar flux from diffusion
 \1 Optimization of QD
   \2 Method that ensures positivity of flux solution
   \2 new diffusion BCs that lead to more accurate solution algorithms
   \2 Analytic forms of QD equation are discretized using positive differencing schemes
     \3 asymptotic exponential decay of discrete solution.  = asymptotic exponential decay of the continuous discrete ordinates solution
     \3 guarantees that angular flux iterate is positive
\end{outline}
\subsection{Discussion}
\begin{outline}
 \1 QD are more rapidly convergent than SI for all meshes. 
 \1 No substantial difference in QD and DD-DSA convergence was observed for diffusive problems
 \1 QD was faster than DSA for coarse meshes
   \2 if DSA used negative flux fixiups
 \1 MQD = diffusion coefficient is modified in QD to exact asymptotic transport eigenvalue is preserved
   \2 is more accurate than DD for coarse grids
\end{outline}
\end{document}
